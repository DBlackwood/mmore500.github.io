\section{Projects and Research Experience}

Otis C. Chapman Honors Thesis | University of Puget Sound, Tacoma, WA \hfill Fall 2016, Spring 2017 \newline
\textit{- Student Researcher} \\
  \vspace{-4.5mm}
 \begin{itemize}
 \item Conduct a review of evolutionary computing literature and synthesize a theoretical analysis of evolvability in collaboration with advisor Dr. America Chambers and reader Dr. Adam Smith
 \item Perform computational experiments with Genetic Regulatory Network models to probe the relationship between phenotypic plasticity and evolvability.
 \item Prepared and delivered general-audience oral presentations at NW Honors Symposium and at the University of Puget Sound.
 \end{itemize}
  \vspace{-3.5mm}

Software Engineering Term Project | University of Puget Sound, Tacoma, WA \hfill Fall 2016 \newline
\textit{- Student Team Member} \\
  \vspace{-4.5mm}
 \begin{itemize}
 \item Collaborated with two other students to develop a full-stack web service leveraging the MEAN.JS framework.
 \item Designed and developed an idea journal service that collects and displays metadata to help users understand where, when, and how they are most creative.
 \end{itemize}
  \vspace{-3.5mm}

Mathematical Biosciences Institute (MBI) Research Experience for Undergraduates | Newark, NJ \hfill Summer 2016 \newline
\textit{- Student Researcher} \\
  \vspace{-4.5mm}
 \begin{itemize}
 \item Designed and numerically evaluated an individual-based set of differential equations to model the foraging behavior of ants over uneven terrain, analyzed predictions of the model over various experimental conditions.
  \item Collaborated with advisors Dr. Jason Graham and Dr. Simon Garnier in the Swarm Lab at the New Jersey Institute of Technology to develop and execute project.
 \item Prepared and delivered oral and poster presentations at a capstone conference in Columbus, Ohio.
 \item Participated in seminars and workshops on mathematical biology coordinated by MBI at The Ohio State University.
 \end{itemize}
  \vspace{-3.5mm}

COMAP Mathematical Contest in Modeling | Tacoma, WA \hfill Spring 2015, 2016, 2017 \newline
\textit{- Contest Participant} \\
\vspace{-4.5mm}
\begin{itemize}
  \item Collaborated in a small team of three students for four days to develop a mathematical model in response to a prompt.
  \item Communicated results in a journal-style paper describing our model and outlining recommendations to policy makers.
  \item In 2017, developed a model of vehicular traffic in the greater Seattle area to assess the impact of self-driving cars on commuter travel delays. Our model predicted that, in certain areas, designating lanes for exclusive use of autonomous would become advantageous once these vehicles constitute approximately 5\% of traffic volume. Our team received a ``Finalist'' designation in the competition, ranking among the top 11 of 1,527 participating teams.
  \item In 2016, developed a model of satellite fragmentation events and the subsequent disbursement of debris in orbit to investigate the feasibility of quick-response efforts to neutralize debris generated by satellite explosions and collisions; our model suggested that, although technically feasible, such efforts would be economically impractical without a significant reduction in launch costs. Our team received received an ``Honorable Mention'' designation in the competition.
  \item In 2015, developed an epidemiological model to investigate the spread of Ebola virus disease and make recommendations on vaccine distribution; our model suggested that regional travel restrictions would not significantly curb the Ebola epidemic in West Africa and that efficient distribution of any vaccination should be prioritized over uniform or widespread distribution.
\end{itemize}
\vspace{-3.5mm}

NASA Undergraduate Research Scholarship | Tacoma, WA \hfill Summer 2015 \newline
\textit{- Student Researcher} \\
  \vspace{-4.5mm}
 \begin{itemize}
 \item Designed, applied for grant funding, and carried out project to develop algorithms for automated extraction of mouse ultrasonic vocalizations from noisy recordings in collaboration with advisor Dr. Adam Smith.
 \item Developed and tested filtering algorithms inspired by the Sobel Edge detection method that, after being trained on human-annotated spectrograms of mouse vocalizations, distinguish between true mouse vocalization signals and background noise, achieving 75\% accuracy at 25\% recall.
 \item Presented results and methodology at a poster session on campus attended by faculty, summer research students, and other students.
 \end{itemize}
  \vspace{-3.5mm}

US Department of Agriculture Horticultural Crops Research Unit | Corvallis, OR  \hfill  June 2013 – Present \newline
\textit{- Biological Science Aide}\\
  \vspace{-4.5mm}
 \begin{itemize}
 \item Collect data for patent applications, perform plant propagation, assist with field maintenance.
 \end{itemize}
  \vspace{-3.5mm}

 John Fowler Laboratory at Oregon State University | Corvallis, OR \hfill Summer 2011, 2012 \newline
\textit{- Laboratory Assistant} \\
 \vspace{-4.5mm}
 \begin{itemize}
 \item Performed experimental inquiry into the role of the exocyst complex in \textit{Arabidopsis thaliana} culminating in a symposium presentation.
 \end{itemize}
  \vspace{-3.5mm}
